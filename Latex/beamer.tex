\documentclass[screen]{beamer}
\usepackage[T1]{fontenc}
\usepackage[latin1]{inputenc}
\usepackage{graphicx}
\usepackage{amssymb}
\usepackage{amsmath}
\usepackage{paralist}
\usepackage[parfill]{parskip}
\usepackage{float}
\usepackage{capt-of}
\usepackage{pdfpages}


% Bruk NTNU-temaet for beamer (her i bokmålvariant), alternativer er
% ntnunynorsk og ntnuenglish.

 
% Angi tittelen, vi gir også en kortere variant som brukes nederst på
% hver slide:
\title[Kuramoto-Sivashinsky]%
{Kuramoto-Sivashinsky equation}

% Datoen blir også trykket på forsida. 
\date{2. April 2014}
%\date{} % Bruk denne hvis du ikke vil ha noe dato på forsida.

% Fra her av begynner selve dokumentet
\begin{document}

% Siden NTNU-malen har en annen bakgrunn på forsida, må dette gjøres
% i en egen kommando, ikke på vanlig beamer-måte:
\begin{frame}
\titlepage
\end{frame}

% Her begynner første slide/frame, (nummer to etter forsida). 
\begin{frame}
\frametitle{Kuramoto-Sivashinsky equation}
\Large
\begin{align*}
u_t + u_{xx} + u_{xxxx} + uu_x = 0 \\
u_t + u_{xx} + u_{xxxx} + \frac{1}{2}(u^2)_x = 0
\end{align*}
\end{frame}


\begin{frame}
\frametitle{About Kuramato-Sivashinsky equation}
\begin{itemize}
\item Stiff
\item Firefronts
\end{itemize}

\end{frame}


\begin{frame}
\frametitle{Forward and central difference}
\begin{align*}
u_t &\approx \frac{\Delta u}{k} = \frac{u^{n+1}-u^n}{k} \\
u_{xx} &\approx \frac{\delta^2 u}{h^2} = \frac{u_{m+1}-2u_{m}+u_{m-1}}{h^2} &= \frac{1}{h^2}Au \\
u_{xxxx} &\approx \frac{\delta^4 u}{h^4} = \frac{u_{m+2}-4u_{m+1}+6u_m-4u_{m-1}+u_{m-2}}{h^4} &= \frac{1}{h^4}A^2u\\
(u^2)_{x} &\approx \frac{\mu \delta u^2}{h} = \frac{(u_{m+1})^2-(u_{m-1})^2}{2h} &= \frac{1}{2h}D\\
\end{align*}
\end{frame}


\begin{frame}
\frametitle{Explicit scheme}
\flushright

\begin{align*}
U^{n+1} = U^n - \frac{k}{h^2}AU^n - \frac{k}{h^4}A^2U^n - \frac{k}{4h}D(U^{n}\odot U^n)
\end{align*}
$\odot =$ Element-wise multiplication
\begin{itemize}
\item Unstable for $k > r \cdot h^4$

\end{itemize}
\end{frame}


\begin{frame}
\frametitle{Crank-Nicolson}
\small
\begin{align*}
 \left[\frac{U^{n+1} - U^n}{k} =
- \frac{1}{2h^2}A(U^{n+1}+U^n) - \frac{1}{2h^4}A^2(U^{n+1}+U^n) \right] - \frac{1}{4h}D(U^{n}\odot U^n)
\end{align*}
\begin{itemize}
\normalsize
\item Not applied to non-linear term
\end{itemize}
\end{frame}


\begin{frame}
\frametitle{Implicit scheme}
\begin{align*}
(I + \frac{k}{2h^2}A + \frac{k}{2h^4}A^2)U^{n+1}
= (I - \frac{k}{2h^2}A - \frac{k}{2h^4}A^2)U^n - \frac{k}{4h}D(U^{n}\odot U^n)
\end{align*}

\begin{itemize}
\item Crank-Nicholson
\item Explicit non-linear term
\item Stable
\end{itemize}

\end{frame}

\begin{frame}
\frametitle{Consistency}
\begin{align*}
u_t= \frac{\Delta u}{k} + O(k) \\
u_{xx} = \frac{\delta^2 u}{h^2} + O(h^2) \\
u_{xxxx} = \frac{\delta^4 u}{h^4} + O(h^2) \\
\mu \delta \left[u(x)^2\right] = \mu\left[u(x+\frac{h}{2}^2) - u(x - \frac{h}{2})^2\right] 
+ O(h^3) \\
uu_{x} = \frac{\mu \delta u^2}{4h} + O(h^2) \\
\end{align*}

\end{frame}


\end{document}







