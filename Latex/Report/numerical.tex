\subsection*{Initial conditions}
In the solution of the KS-equation we had periodic boundary conditions, i.e. $u(0,t) = u(L,t)$. We also used L-periodic initial conditions. We experienced that a common initial condition used in several other reports was

\begin{equation}
\label{initialCondition}
u(x,0) = \cos(\frac{x}{16})(1 + \sin(\frac{x}{16}).
\end{equation}

We also tried the initial condition

\begin{equation}
\label{initialCondition2}
u(x,0) = \frac{1}{\sqrt{2}} \sin(x) - \frac{1}{8}\sin(2x),
\end{equation}

which worked well. The L-periodic initial conditions is customarily taken \cite{periodicInitial} to satisfy

\begin{equation}
\int_0^L\! f(x)\,\textrm{d}x = 0,
\end{equation}
which both of our initial conditions satisfy. The same article also states that for L-periodic initial data, a unique solution for \eqref{KSeq} exits, and is bounded as $t\rightarrow\infty$. The bound has been proven to be smaller than $O(L^{8/5})$. In our numerical tests, with $t=5000$, the initial condition \eqref{initialCondition} did indeed not exceed the bound, nor did \eqref{initialCondition2}.


