The truncation error imposed by the discretization in space is proven by Taylor expansions. Using the central difference operator on itself, gives rise to the following expression,

\begin{align*}
\delta \delta u = \delta (u_{m+\frac{1}{2}}^n - u_{m-\frac{1}{2}}^ {n}) \\
= u_{m+1}^n -2u_m^n + u_{m-1}^n,
\end{align*}
and by Taylor expanding the terms,

\begin{align*}
u_{m \pm 1}^n = u_m^n \pm h\partial_xu_m^n + \frac{h^2}{2}\partial_x^2u_m^n \pm \frac{h^3}{6}\partial_x^3u_m^n + O(h^4).
\end{align*}
This leads to

\begin{align*}
\delta^2 u_m^n = \frac{h^2}{2}\partial_x^2 u_m^n + O(h^4),
\end{align*}
and it is seen that the $\delta^2$-operator can be used to approximate the $\partial^2$-operator. The discretization of the $u_{xxxx}$-term is derived the same way, but with the $\delta^4$-operator approximating the $\partial^4$-operator:

\begin{align*}
\delta^4 u_m^n = \frac{h^4}{2}\partial_x^4 u_m^n + O(h^6)
\end{align*}
The spatial discretization of the $uu_{x}$-term is derived by defining $v(x,t) = u(x,t)^2$. Using the central difference on $v_{m}^n$,

\begin{align*}
\delta v_m^n = v_{m+\frac{1}{2}}^n - v_{m-\frac{1}{2}}^n,
\end{align*}
and then the mean operator $\mu$,
\begin{align*}
\mu\delta v_m^n = \frac{1}{2}(v_{m+1}^n - v_{m-1}^n),
\end{align*}
gives a discretization for the non-linear term. It is seen from Taylor expanding the terms that,

\begin{align*}
v_{m \pm 1}^n = v_m^n \pm h\partial_xv_m^n + \frac{h^2}{2}\partial_x^2v_m^n \pm \frac{h^3}{6}\partial_x^3v_m^n + O(h^4), 
\end{align*}
which by the substitution $2u\partial_xu = \partial_xv$ gives
\begin{align*}
v_{m \pm 1}^n = (u_m^n)^2 \pm 2hu_m^n\partial_xu_m^n + \frac{h^2}{2}\partial_x^2(u_m^n)^2 \pm \frac{h^3}{6}\partial_x^3(u_m^n)^2 + O(h^4), \\
\mu \delta v_m^n = 2hu_m^n\partial_xu_m^n + \frac{h^3}{6}\partial_{x}^3(v_m^n) + O(h^5).
\end{align*}

By the derivations above, we have the following discretizations with errors in space:
\begin{align*}
u_{xx} = \frac{\delta^2 u}{h^2} - \frac{h^2}{12}\partial_{x}^2u_m^n + O(h^4) \\
u_{xxxx} = \frac{\delta^4 u}{h^4} - \frac{h^2}{6}\partial_{x}^6u_m^n + O(h^4) \\
uu_{x} = \frac{\mu \delta u^2}{2h} - \frac{h^2}{12}\partial_{x}^3(u_m^n)^2 + O(h^4)
\end{align*}
That is, each term gives a contribution of $O(h^2)$ to the local truncation error.

For the explicit method, forward difference was used to discretize in time:
\begin{align*}
\Delta u = u_m^{n+1}-u_m^n
\end{align*}
And by Taylor expanding,
\begin{align*}
u_m^{n+1} = u_m^n + k\partial_tu_m^n + \frac{k^2}{2}\partial_t^2u_m^n + O(k^3),
\end{align*}
an expression for the time derivative is seen to be:
\begin{align*}
u_t = \frac{\Delta u}{k} - \frac{k}{2}\partial_t^2u_m^n + O(k^2)
\end{align*}
That is, a contribution of $O(k)$ to the local truncation error.

For the implicit method, the trapezoidal rule is used to discretize in time:
\begin{align*}
u_{m}^{n+1} - u_m^n = \int\limits_{t_{n}}^{t_{n+1}} u_{t}(x_{m},t_{n})\, \mathrm{d}t = \frac{k}{2}(u_t(x_m,t_n) + u_t(x_m,t_{n+1})) - \frac{k^3}{12}u_{ttt}(x_m,t_{n+\frac{1}{2}}) + O(k^5)
\end{align*}

By using the space discretizations derived above in the approximation for the derivative in time, an expression with a non-linear term for the time step $t=t_{n+1}$ is derived. This non-linear implicit term is approximated by a first order Taylor expansion around $t=t_{n}$:
\begin{align*}
(u_m^{n+1})^2 = (u_m^n)^2 + 2ku_m^n\partial_tu_m^n + O(k^2).
\end{align*}
That is, a contribution of $O(k)$ to the local truncation error.

By the above expressions we finally arrive at the conclusion of this section. The local truncation error,$\tau_m^n$, for both the explicit and the implicit method is of same order in both time and space:

\begin{align*}
\tau_m^n = O(k + h^2)
\end{align*}
Since $\tau_m^n \xrightarrow{k,h \to 0} 0$, we conclude that both the explicit and the implicit methods are consistent of order 1 in time and 2 in space.

%hei%