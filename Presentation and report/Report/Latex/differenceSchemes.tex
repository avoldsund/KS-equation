\subsection{Explicit method}
The KS-equation can be approximated by using central differences in space, and forward differences in time. Notice that the mean operator was used on the nonlinear term to avoid half space steps.    The nonlinear term has also been rewritten from $uu_x$ to $\frac{1}{2}(u^2)_x$.
\begin{equation*}
\begin{aligned}
u_t &\approx \frac{\Delta u}{k} = \frac{u^{n+1}-u^n}{k} \\
u_{xx} &\approx \frac{\delta^2 u}{h^2} = \frac{u_{m+1}-2u_{m}+u_{m-1}}{h^2} = \frac{1}{h^2}Au \\
u_{xxxx} &\approx \frac{\delta^4 u}{h^4} = \frac{u_{m+2}-4u_{m+1}+6u_m-4u_{m-1}+u_{m-2}}{h^4} = \frac{1}{h^4}A^2u\\
(u^2)_{x} &\approx \frac{\mu \delta u^2}{h} = \frac{(u_{m+1})^2-(u_{m-1})^2}{2h} = \frac{1}{2h}D\\
\end{aligned}
\end{equation*}

which leads to the system

\begin{equation}
\label{explDiff}
U^{n+1} = (I - \frac{k}{h^2}A - \frac{k}{h^4}A^2)U^n - \frac{k}{4h}D(U^{n}\odot U^n).
\end{equation}

\subsection{Implicit-explicit method}
Applying the trapezoidal rule to all linear terms gives a Crank-Nicolson-type method which is implicit-explicit. The nonlinear term is the same as in the explicit method \eqref{explDiff}.
\begin{align*}
(I + \frac{k}{2h^2}A + \frac{k}{2h^4}A^2)U^{n+1}
= (I - \frac{k}{2h^2}A - \frac{k}{2h^4}A^2)U^n - \frac{k}{4h}D(U^{n}\odot U^n)
\end{align*}

%a

