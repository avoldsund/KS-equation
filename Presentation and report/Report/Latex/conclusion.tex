By what we have seen from the numerical experiments, when comparing the two difference schemes, the implicit scheme shows better properties regarding stability than the explicit scheme. For the latter a stability criterion for the time step, $k/h^4 \le 1/8$, was given under some assumptions, but this time step is very low and unpractical. For the implicit method, we were not able to give any analytic reasonable criterion, but numerical testing implied stability for time steps smaller than $\frac{1}{3}$. Though the numerical computations are much more time consuming for the implicit method, due to the stiffness and instability of the system, this method would serve better in practice for modeling this equation.

When it comes to the collaboration in the group, it has been a real delight. We have known each other for a very long time and have worked together on several projects prior to this project. Even though we experienced trouble with the equation at times, we overcame the obstacles as a team, and both helped, and learned from each other. We split the workload in equal parts, and worked together the entire time. We feel that our interest in numerical mathematics has increased during the project, and we feel that we have learned a lot during the last couple of months.